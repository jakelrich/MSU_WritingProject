% Copyright 2004 by Till Tantau <tantau@users.sourceforge.net>.
%
% In principle, this file can be redistributed and/or modified under
% the terms of the GNU Public License, version 2.
%
% However, this file is supposed to be a template to be modified
% for your own needs. For this reason, if you use this file as a
% template and not specifically distribute it as part of a another
% package/program, I grant the extra permission to freely copy and
% modify this file as you see fit and even to delete this copyright
% notice. 

\documentclass{beamer}\usepackage[]{graphicx}\usepackage[]{color}
%% maxwidth is the original width if it is less than linewidth
%% otherwise use linewidth (to make sure the graphics do not exceed the margin)
\makeatletter
\def\maxwidth{ %
  \ifdim\Gin@nat@width>\linewidth
    \linewidth
  \else
    \Gin@nat@width
  \fi
}
\makeatother

\definecolor{fgcolor}{rgb}{0.345, 0.345, 0.345}
\newcommand{\hlnum}[1]{\textcolor[rgb]{0.686,0.059,0.569}{#1}}%
\newcommand{\hlstr}[1]{\textcolor[rgb]{0.192,0.494,0.8}{#1}}%
\newcommand{\hlcom}[1]{\textcolor[rgb]{0.678,0.584,0.686}{\textit{#1}}}%
\newcommand{\hlopt}[1]{\textcolor[rgb]{0,0,0}{#1}}%
\newcommand{\hlstd}[1]{\textcolor[rgb]{0.345,0.345,0.345}{#1}}%
\newcommand{\hlkwa}[1]{\textcolor[rgb]{0.161,0.373,0.58}{\textbf{#1}}}%
\newcommand{\hlkwb}[1]{\textcolor[rgb]{0.69,0.353,0.396}{#1}}%
\newcommand{\hlkwc}[1]{\textcolor[rgb]{0.333,0.667,0.333}{#1}}%
\newcommand{\hlkwd}[1]{\textcolor[rgb]{0.737,0.353,0.396}{\textbf{#1}}}%
\let\hlipl\hlkwb

\usepackage{framed}
\makeatletter
\newenvironment{kframe}{%
 \def\at@end@of@kframe{}%
 \ifinner\ifhmode%
  \def\at@end@of@kframe{\end{minipage}}%
  \begin{minipage}{\columnwidth}%
 \fi\fi%
 \def\FrameCommand##1{\hskip\@totalleftmargin \hskip-\fboxsep
 \colorbox{shadecolor}{##1}\hskip-\fboxsep
     % There is no \\@totalrightmargin, so:
     \hskip-\linewidth \hskip-\@totalleftmargin \hskip\columnwidth}%
 \MakeFramed {\advance\hsize-\width
   \@totalleftmargin\z@ \linewidth\hsize
   \@setminipage}}%
 {\par\unskip\endMakeFramed%
 \at@end@of@kframe}
\makeatother

\definecolor{shadecolor}{rgb}{.97, .97, .97}
\definecolor{messagecolor}{rgb}{0, 0, 0}
\definecolor{warningcolor}{rgb}{1, 0, 1}
\definecolor{errorcolor}{rgb}{1, 0, 0}
\newenvironment{knitrout}{}{} % an empty environment to be redefined in TeX

\usepackage{alltt}

% There are many different themes available for Beamer. A comprehensive
% list with examples is given here:
% http://deic.uab.es/~iblanes/beamer_gallery/index_by_theme.html
% You can uncomment the themes below if you would like to use a different
% one:
\usetheme{AnnArbor}
%\usetheme{Antibes}
%\usetheme{Bergen}
%\usetheme{Berkeley}
%\usetheme{Berlin}
%\usetheme{Boadilla}
%\usetheme{boxes}
%\usetheme{CambridgeUS}
%\usetheme{Copenhagen}
%\usetheme{Darmstadt}
%\usetheme{default}
%\usetheme{Frankfurt}
%\usetheme{Goettingen}
%\usetheme{Hannover}
%\usetheme{Ilmenau}
%\usetheme{JuanLesPins}
%\usetheme{Luebeck}
%\usetheme{Madrid}
%\usetheme{Malmoe}
%\usetheme{Marburg}
%\usetheme{Montpellier}
%\usetheme{PaloAlto}
%\usetheme{Pittsburgh}
%\usetheme{Rochester}
%\usetheme{Singapore}
%\usetheme{Szeged}
%\usetheme{Warsaw}

\usepackage{amsmath}

\title{Comparison of GLMM Estimation Methods}

% A subtitle is optional and this may be deleted

\author{Jacob Rich}
% - Give the names in the same order as the appear in the paper.
% - Use the \inst{?} command only if the authors have different
%   affiliation.

%\institute[Montana State University] % (optional, but mostly needed)
%{
  %\inst{1}%
 % Department of Mathematical Sciences\\
  %Montana State University}
% - Use the \inst command only if there are several affiliations.
% - Keep it simple, no one is interested in your street address.

\date{April 10th, 2018}
% - Either use conference name or its abbreviation.
% - Not really informative to the audience, more for people (including
%   yourself) who are reading the slides online

%\subject{Theoretical Computer Science}
% This is only inserted into the PDF information catalog. Can be left
% out. 

% If you have a file called "university-logo-filename.xxx", where xxx
% is a graphic format that can be processed by latex or pdflatex,
% resp., then you can add a logo as follows:

% \pgfdeclareimage[height=0.5cm]{university-logo}{university-logo-filename}
% \logo{\pgfuseimage{university-logo}}

% Delete this, if you do not want the table of contents to pop up at
% the beginning of each subsection:
%\AtBeginSubsection[]
%{
%  \begin{frame}<beamer>{Outline}
%    \tableofcontents[currentsection,currentsubsection]
%end{frame}
%}

% Let's get started
\IfFileExists{upquote.sty}{\usepackage{upquote}}{}
\begin{document}

\begin{frame}
  \titlepage
\end{frame}

%\begin{frame}{Outline}
  %\tableofcontents
  % You might wish to add the option [pausesections]
%\end{frame}


%    \begin{block}{Block Title}
%    You can also highlight sections of your presentation in a block, with it's own title
%    \end{block}

% Section and subsections will appear in the presentation overview
% and table of contents.
\section{Background}

\begin{frame}{Generalized Linear Mixed Models}
  \begin{itemize}
        \item In ecological and evolutionary biology, ordinary linear models are not always well suited for data analysis.
        \begin{itemize}
          \item Presence/absence and count data are two common situations where a linear model is not well suited (non-normal responses).
          \item Dependence between observations occur through repeated measures, clustered observations, or within spatial elements violate independence assumptions (correlated observations).
        \end{itemize}
        \item If both non-normal responses and correlated observations occur in a dataset, we can use generalized linear mixed models (GLMMs) for estimation and inference.
  \end{itemize}
\end{frame}

\begin{frame}{A Motivating Example}
  \begin{itemize}
        \item Owlet begging data from Roulin and Bersier (2007) is an example of a dataset suitable for analysis with GLMMs.
        \begin{itemize}
          \item Data consist of 599 observations from 27 barn owl nests in western Switzerland.
          \item Response: number of calls in a 30 sec. interval before the parent arrived. 
          \item Covariates used: brood size, food treatment, and arrival time of parent (between 2130 and 0530 hours).
          \item Nest is treated as a random effect. 
        \end{itemize}
        \item Observations are correlated at the nest level due to repeated measurements and the count data should follow a Poisson distribution. 
  \end{itemize}
\end{frame}

\begin{frame}{Issues with Estimation}
	 \begin{block}{Problem:}
      The likelihood functions of GLMMs involve high-dimensional integrals that lack closed form solutions, making evaluation of the exact likelihood function essentially impossible. 
   \end{block}
	
	\begin{itemize}
		\item In place of closed form solutions, a number of approximation methods have been developed since the introduction of GLMMs by McCullagh and Nelder (1989).           Commonly used approximations include:
    	\begin{itemize}
    		\item Penalized Quasi-likelihood
    		\item Laplace Approximation
    		\item (Adaptive) Gaussian Hermite Quadrature
    	\end{itemize}
		\item Bayesian approaches to GLMMs are also commonly used as it sidesteps the issues integration issues involved with the exact likelihood.
	\end{itemize}
\end{frame}

\begin{frame}{GLMM Estimation Methods}
	

\end{frame}

%\begin{figure}
%	\includegraphics[scale=0.45]{Correlationexamples2.png}
%\end{figure}

\end{document}
